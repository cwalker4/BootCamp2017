\documentclass[letterpaper,12pt]{article}
\usepackage{array}
\usepackage{threeparttable}
\usepackage{geometry}
\geometry{letterpaper,tmargin=1in,bmargin=1in,lmargin=1.25in,rmargin=1.25in}
\usepackage{fancyhdr,lastpage}
\pagestyle{fancy}
\lhead{}
\chead{}
\rhead{}
\lfoot{}
\cfoot{}
\rfoot{\footnotesize\textsl{Page \thepage\ of \pageref{LastPage}}}
\renewcommand\headrulewidth{0pt}
\renewcommand\footrulewidth{0pt}
\usepackage[format=hang,font=normalsize,labelfont=bf]{caption}
\usepackage{listings}
\lstset{frame=single,
  language=Python,
  showstringspaces=false,
  columns=flexible,
  basicstyle={\small\ttfamily},
  numbers=none,
  breaklines=true,
  breakatwhitespace=true
  tabsize=3
}
\usepackage{amsmath}
\usepackage{amssymb}
\usepackage{amsthm}
\usepackage{harvard}
\usepackage{setspace}
\usepackage{float,color}
\usepackage[pdftex]{graphicx}
\usepackage{hyperref}
\hypersetup{colorlinks,linkcolor=red,urlcolor=blue}
\theoremstyle{definition}
\newtheorem{theorem}{Theorem}
\newtheorem{acknowledgement}[theorem]{Acknowledgement}
\newtheorem{algorithm}[theorem]{Algorithm}
\newtheorem{axiom}[theorem]{Axiom}
\newtheorem{case}[theorem]{Case}
\newtheorem{claim}[theorem]{Claim}
\newtheorem{conclusion}[theorem]{Conclusion}
\newtheorem{condition}[theorem]{Condition}
\newtheorem{conjecture}[theorem]{Conjecture}
\newtheorem{corollary}[theorem]{Corollary}
\newtheorem{criterion}[theorem]{Criterion}
\newtheorem{definition}[theorem]{Definition}
\newtheorem{derivation}{Derivation} % Number derivations on their own
\newtheorem{example}[theorem]{Example}
\newtheorem{exercise}[theorem]{Exercise}
\newtheorem{lemma}[theorem]{Lemma}
\newtheorem{notation}[theorem]{Notation}
\newtheorem{problem}[theorem]{Problem}
\newtheorem{proposition}{Proposition} % Number propositions on their own
\newtheorem{remark}[theorem]{Remark}
\newtheorem{solution}[theorem]{Solution}
\newtheorem{summary}[theorem]{Summary}
%\numberwithin{equation}{section}
\bibliographystyle{aer}
\newcommand\ve{\varepsilon}
\newcommand\boldline{\arrayrulewidth{1pt}\hline}


\begin{document}

\begin{flushleft}
  \textbf{\large{Math Problem Set \#5}} \\
  Charlie Walker
\end{flushleft}

\vspace{5mm}

\noindent\textbf{7.1}\\
Proof by induction. For $k=2$, the set of all combinations of the form $\lambda_1x_1 + \lambda_2x_2$, where $\lambda_1, \lambda_2 \geq 0$ and $\lambda_1 + \lambda_2 = 1$ is clearly convex, by the definition of a convex set. For $k=3$

\noindent\textbf{7.2}\\
(i) Let $x, y$ be two vectors in the hyperplane. Since each vector is in the hyperplane, $a \cdot x = a \cdot y = 0$. Convexity requires $\lambda x + (1 - \lambda) y$ to be in the hyperplane; in other words,
\begin{align*}
&a \cdot (\lambda x + (1 - \lambda) y) = 0\\
&= a \lambda x + a(1-\lambda)y\\
&= \lambda ax + (1 - \lambda) ay
\end{align*}
Since $a \cdot x = a \cdot y = 0$, the above expression is equal to zero, implying that the hyperplane is convex.\\

\noindent (ii) Halfspaces are the set of vectors $x$ such that $a \cdot x \leq 0$. As above, take any two vectors $x, y$ in the halfspace. 
\begin{align*}
a \cdot (\lambda x + (1-\lambda)y &= a\lambda x + a(1-\lambda)y\\
&= a \lambda x + a(1-\lambda) y\\
&= \lambda ax + (1-\lambda)ay
\end{align*}
Since $a \cdot x, a\cdot y$ are less than or equal to zero, the above expression is less than or equal to zero, implying that the halfspace is convex.\\

\noindent\textbf{7.4}\\









\end{document}