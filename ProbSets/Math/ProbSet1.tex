\documentclass[letterpaper,12pt]{article}
\usepackage{array}
\usepackage{threeparttable}
\usepackage{geometry}
\geometry{letterpaper,tmargin=1in,bmargin=1in,lmargin=1.25in,rmargin=1.25in}
\usepackage{fancyhdr,lastpage}
\pagestyle{fancy}
\lhead{}
\chead{}
\rhead{}
\lfoot{}
\cfoot{}
\rfoot{\footnotesize\textsl{Page \thepage\ of \pageref{LastPage}}}
\renewcommand\headrulewidth{0pt}
\renewcommand\footrulewidth{0pt}
\usepackage[format=hang,font=normalsize,labelfont=bf]{caption}
\usepackage{listings}
\lstset{frame=single,
  language=Python,
  showstringspaces=false,
  columns=flexible,
  basicstyle={\small\ttfamily},
  numbers=none,
  breaklines=true,
  breakatwhitespace=true
  tabsize=3
}
\usepackage{amsmath}
\usepackage{amssymb}
\usepackage{amsthm}
\usepackage{harvard}
\usepackage{setspace}
\usepackage{float,color}
\usepackage[pdftex]{graphicx}
\usepackage{hyperref}
\hypersetup{colorlinks,linkcolor=red,urlcolor=blue}
\theoremstyle{definition}
\newtheorem{theorem}{Theorem}
\newtheorem{acknowledgement}[theorem]{Acknowledgement}
\newtheorem{algorithm}[theorem]{Algorithm}
\newtheorem{axiom}[theorem]{Axiom}
\newtheorem{case}[theorem]{Case}
\newtheorem{claim}[theorem]{Claim}
\newtheorem{conclusion}[theorem]{Conclusion}
\newtheorem{condition}[theorem]{Condition}
\newtheorem{conjecture}[theorem]{Conjecture}
\newtheorem{corollary}[theorem]{Corollary}
\newtheorem{criterion}[theorem]{Criterion}
\newtheorem{definition}[theorem]{Definition}
\newtheorem{derivation}{Derivation} % Number derivations on their own
\newtheorem{example}[theorem]{Example}
\newtheorem{exercise}[theorem]{Exercise}
\newtheorem{lemma}[theorem]{Lemma}
\newtheorem{notation}[theorem]{Notation}
\newtheorem{problem}[theorem]{Problem}
\newtheorem{proposition}{Proposition} % Number propositions on their own
\newtheorem{remark}[theorem]{Remark}
\newtheorem{solution}[theorem]{Solution}
\newtheorem{summary}[theorem]{Summary}
%\numberwithin{equation}{section}
\bibliographystyle{aer}
\newcommand\ve{\varepsilon}
\newcommand\boldline{\arrayrulewidth{1pt}\hline}


\begin{document}

\begin{flushleft}
  \textbf{\large{Problem Set \#1}} \\
  MACS 30000, Dr. Evans \\
  Charlie Walker
\end{flushleft}
\vspace{5mm}
\noindent\textbf{Problem 1} \\

\noindent\textbf{3.6}

\noindent By independence,
\begin{equation*}
\sum_{i \in I}P(A \cap B_i) = P(A)[\sum_{i \in I}P(B_i)] 
\end{equation*}
By additivity,
\begin{equation*}
P(A)[\sum_{i \in I}P(B_i)] = P(A)P(\bigcup_{i \in I} B_i)
\end{equation*}
Since $\bigcup_{i \in I} B_i= \Omega$ and $B_i \cap B_j = \emptyset$ for all  $i \neq j$, $P(\bigcup_{i \in I} B_i) = 1 \implies P(A)P(\bigcup_{i \in I} B_i) = P(A)$, as desired.\newline\\
\noindent\textbf{3.8}\\
\noindent
\begin{equation*}
\begin{split}
1 - \prod^{n}_{k=1}(1 - P(E_k)) & = 1 - \prod^{n}_{k=1}P(E^{c}_{k}) \\
& = 1 - P(\bigcap^{n}_{k=1}E^{c}_{k})\\
& = P((\bigcap^{n}_{k=1}E^{c}_{k})^{c}) \\
& = P(\bigcup^{n}_{k=1}E_k) 
\end{split}
\end{equation*}
Where line (2) follows from independence, and line (4) by DeMorgan's Laws.\newline\\
\noindent\textbf{3.11}\\
\noindent Assume that the DNA test is perfectly accurate in identifying someone who was at the crime scene, i.e. $P(pos | guilty) = 1$. From the description, we have:
\begin{align*}
P(guilty) &= \frac{1}{250m} \\
P(pos) &= \frac{1}{3m} \\
\end{align*}
We want the probability that an individual was at the crime scene, given their DNA test was positive, that is $P(guilty | positive)$. Bayes' Rule gives:
\begin{equation*}
\begin{split}
P(guilty | pos) &= \frac{P(pos | guilty)P(guilty)}{P(pos)}\\
&\approx 1.2\%
\end{split}
\end{equation*}\newline\\
\noindent\textbf{3.11}\\
\noindent First, some setup. Define:\par
	D1 = Morty opens door 1\par
	D2 = Morty opens door 2\par
	D3 = Morty opens door 3\par
	C1 = Car behind door 1\par
	C2 = Car behind door 2\par
	C3 = Car behind door 3\newline\\
\noindent Assume you choose door 1. Then,
\begin{align*}
P(D3 | C1) &= P(D2 | C1) = \frac{1}{2}\\
P(D3 | C3) &= 0\\
P(D3 | C2) &= 1\\
\end{align*}
And,
\begin{equation*}
\begin{split}
P(C1 | D3) &= \frac{P(D3 | C1)P(C1)}{P(D3)}\\
&= \frac{1}{3}\\
P(C2 | D3) &= \frac{P(D3 | C2)P(C2)}{P(D3)}\\
&= \frac{2}{3}
\end{split}
\end{equation*}
For the 10-door case, if Monty opens 8 doors, the contestant has a $\frac{9}{10}$ chance of winning if they switch, and a $\frac{1}{10}$ chance if they do not.
\end{document}

