\documentclass[letterpaper,12pt]{article}
\usepackage{array}
\usepackage{threeparttable}
\usepackage{geometry}
\geometry{letterpaper,tmargin=1in,bmargin=1in,lmargin=1.25in,rmargin=1.25in}
\usepackage{fancyhdr,lastpage}
\pagestyle{fancy}
\lhead{}
\chead{}
\rhead{}
\lfoot{}
\cfoot{}
\rfoot{\footnotesize\textsl{Page \thepage\ of \pageref{LastPage}}}
\renewcommand\headrulewidth{0pt}
\renewcommand\footrulewidth{0pt}
\usepackage[format=hang,font=normalsize,labelfont=bf]{caption}
\usepackage{listings}
\lstset{frame=single,
  language=Python,
  showstringspaces=false,
  columns=flexible,
  basicstyle={\small\ttfamily},
  numbers=none,
  breaklines=true,
  breakatwhitespace=true
  tabsize=3
}
\usepackage{amsmath}
\usepackage{amssymb}
\usepackage{amsthm}
\usepackage{harvard}
\usepackage{setspace}
\usepackage{float,color}
\usepackage[pdftex]{graphicx}
\usepackage{hyperref}
\hypersetup{colorlinks,linkcolor=red,urlcolor=blue}
\theoremstyle{definition}
\newtheorem{theorem}{Theorem}
\newtheorem{acknowledgement}[theorem]{Acknowledgement}
\newtheorem{algorithm}[theorem]{Algorithm}
\newtheorem{axiom}[theorem]{Axiom}
\newtheorem{case}[theorem]{Case}
\newtheorem{claim}[theorem]{Claim}
\newtheorem{conclusion}[theorem]{Conclusion}
\newtheorem{condition}[theorem]{Condition}
\newtheorem{conjecture}[theorem]{Conjecture}
\newtheorem{corollary}[theorem]{Corollary}
\newtheorem{criterion}[theorem]{Criterion}
\newtheorem{definition}[theorem]{Definition}
\newtheorem{derivation}{Derivation} % Number derivations on their own
\newtheorem{example}[theorem]{Example}
\newtheorem{exercise}[theorem]{Exercise}
\newtheorem{lemma}[theorem]{Lemma}
\newtheorem{notation}[theorem]{Notation}
\newtheorem{problem}[theorem]{Problem}
\newtheorem{proposition}{Proposition} % Number propositions on their own
\newtheorem{remark}[theorem]{Remark}
\newtheorem{solution}[theorem]{Solution}
\newtheorem{summary}[theorem]{Summary}
%\numberwithin{equation}{section}
\bibliographystyle{aer}
\newcommand\ve{\varepsilon}
\newcommand\boldline{\arrayrulewidth{1pt}\hline}


\begin{document}

\begin{flushleft}
  \textbf{\large{Problem Set \#1}} \\
  Charlie Walker
\end{flushleft}
\vspace{5mm}
\noindent\textbf{Problem 1} \\

\noindent\textbf{3.6}

\noindent By independence,
\begin{equation*}
\sum_{i \in I}P(A \cap B_i) = P(A)[\sum_{i \in I}P(B_i)] 
\end{equation*}
By additivity,
\begin{equation*}
P(A)[\sum_{i \in I}P(B_i)] = P(A)P(\bigcup_{i \in I} B_i)
\end{equation*}
Since $\bigcup_{i \in I} B_i= \Omega$ and $B_i \cap B_j = \emptyset$ for all  $i \neq j$, $P(\bigcup_{i \in I} B_i) = 1 \implies P(A)P(\bigcup_{i \in I} B_i) = P(A)$, as desired.\newline\\
\noindent\textbf{3.8}\\
\noindent
\begin{equation*}
\begin{split}
1 - \prod^{n}_{k=1}(1 - P(E_k)) & = 1 - \prod^{n}_{k=1}P(E^{c}_{k}) \\
& = 1 - P(\bigcap^{n}_{k=1}E^{c}_{k})\\
& = P((\bigcap^{n}_{k=1}E^{c}_{k})^{c}) \\
& = P(\bigcup^{n}_{k=1}E_k) 
\end{split}
\end{equation*}
Where line (2) follows from independence, and line (4) by DeMorgan's Laws.\newline\\
\noindent\textbf{3.11}\\
\noindent Assume that the DNA test is perfectly accurate in identifying someone who was at the crime scene, i.e. $P(pos | guilty) = 1$. From the description, we have:
\begin{align*}
P(guilty) &= \frac{1}{250m} \\
P(pos) &= \frac{1}{3m} \\
\end{align*}
We want the probability that an individual was at the crime scene, given their DNA test was positive, that is $P(guilty | positive)$. Bayes' Rule gives:
\begin{equation*}
\begin{split}
P(guilty | pos) &= \frac{P(pos | guilty)P(guilty)}{P(pos)}\\
&\approx 1.2\%
\end{split}
\end{equation*}\newline\\
\noindent\textbf{3.12}\\
\noindent First, some setup. Define the events:\par
	D1 = Morty opens door 1\par
	D2 = Morty opens door 2\par
	D3 = Morty opens door 3\par
	C1 = Car behind door 1\par
	C2 = Car behind door 2\par
	C3 = Car behind door 3\newline\\
\noindent Assume you choose door 1. Then,
\begin{align*}
P(D3 | C1) &= P(D2 | C1) = \frac{1}{2}\\
P(D3 | C3) &= 0\\
P(D3 | C2) &= 1\\
\end{align*}
And,
\begin{equation*}
\begin{split}
P(C1 | D3) &= \frac{P(D3 | C1)P(C1)}{P(D3)}\\
&= \frac{1}{3}\\
P(C2 | D3) &= \frac{P(D3 | C2)P(C2)}{P(D3)}\\
&= \frac{2}{3}
\end{split}
\end{equation*}
For the 10-door case, if Monty opens 8 doors, the contestant has a $\frac{9}{10}$ chance of winning if they switch, and a $\frac{1}{10}$ chance if they do not.\newline\\
\noindent\textbf{3.16}
\noindent
\begin{align*}
E[(X-\mu)^2)] &= E[X^2 - 2X\mu + \mu^2]\\
&= E[X^2] - 2E[X]E[X] + E[X]^2\\
&= E[X^2] - E[X]^2
\end{align*}\newline\\
\noindent\textbf{3.33}\\
\noindent Chebyshev's Inequality for a binomial random variable with mean $np$ and variance $(1-p)p$ gives:
\begin{equation*}
P(\left|B-np\right| \geq \epsilon) \leq \frac{(1-p)p}{\epsilon^2}\\
\implies P\Bigg(\left|{\frac{B}{n}-p}\right| \geq \epsilon\Bigg) \leq \frac{(1-p)p}{n\epsilon^2}
\end{equation*}\newline\\
\noindent\textbf{3.36}\\
\noindent Let $B_n$ denote individual trials, $B_n \in \{0, 1\}$. Each enrolment event is a Bernoulli trial with success probability $P(B_n = 1) = 0.801$. Define $X_n = B_1 + B_2 + ... + B_n$. For $n = 6242, E[X] = (6242)(0.801) = 5000, Var[X] = (6242)(0.801)(0.199) = 994.97$. By the central limit theorem, $X \sim N(5000, 994.97)$. 
\begin{align*}
P(X > 5500) &= P\Bigg(\frac{X-5000}{\sqrt{994.97}} > \frac{500}{\sqrt{994.97}}\Bigg)\\
&= 1 - \Phi\Bigg(\frac{500}{\sqrt{994.97}}\Bigg)\\
&= 1 - \Phi(15.85)\\
&\approx 0.003\%
\end{align*}\newline\\
\noindent\textbf{Problem 2}\\
\noindent (a) Throw two 6-sided die. Define events:\par
A: sum of points is 7\par
B: die \#1 is 3\par
C: die \#2 is 4\\
Then,
\begin{align*}
P(A) &= P(B) = P(C) = \frac{1}{6}\\
P(A \cap B) &= P(A \cap C) = P(B \cap C) = \frac{1}{36}\\
P(A \cap B \cap C) &= \frac{1}{6} \ne P(A)P(B)P(C) = \frac{1}{216}
\end{align*}\\
(b)\newline\\
\noindent\textbf{Problem 3}\\
Benford's Law states that $P(d) = log_{10}(1+\frac{1}{d}), d \in \{1, 2, ... , 9\}. \sum_{i=1}^{9}log_{10}(1+\frac{1}{d}) = log(2) + log(\frac{3}{2}) + log(\frac{4}{3}) + ... + log(\frac{10}{9}) = 1 \implies$ Benford's Law is a well-defined discrete probability distribution.\newline\\
\noindent\textbf{Problem 4}\\
(a) \begin{align*}
E[X] &= \sum_{n=1}^{\infty}2^nP(i)\\
&= \sum_{n=1}^{\infty}2^n\Big(\frac{1}{2}\Big)\Big(\frac{1}{2}\Big)^{n-1}\\
&= \frac{1}{2}\cdot2+\frac{1}{4}\cdot4+\frac{1}{8}\cdot8+...\\
&= 1+1+1+...\\
&=+\infty
\end{align*}\\
(b) \begin{align*}
E[\ln X] &= \sum_{n=1}^{\infty}\ln(2^n)P(n)\\
&= \sum_{n=1}^{\infty}\ln(2^n)\Big(\frac{1}{2}\Big)\Big(\frac{1}{2}\Big)^{n-1}\\
&= \frac{\ln(2)}{2}\sum_{n=1}^{\infty}n\Big(\frac{1}{2}\Big)^{n-1}\\
&= 2\ln(2)
\end{align*}\newline\\
\noindent\textbf{Problem 5}\\
Both investors should invest in the foreign currency:
\begin{align*}
E[USD / CHF] &= (0.5)(1.25) + (0.5)(0.8) = 1.025\\
E[CHF / USD] &= (0.5)(0.8) + (0.5)(1.25) = 1.025
\end{align*}\newline\\
\noindent\textbf{Problem 6}\\
(a)\\
(b)\\
(c) Three standard normal distributions satisfy the conditions, i.e. $X\sim N(0,1), Y \sim N(0,1), Z \sim (0,1)$\\


\noindent\textbf{Problem 7}\\
(a) \begin{align*}
E[Y] &= E[XZ] = E[X]E[Z] = 0\\
Var[Y] &= Var[XZ]\\
&= E[XZ^2] - E[XZ]^2\\
&= E[X^2]E[Z^2] - E[X]^2E[Z]^2\\
&= E[X^2]E[Z^2]\\
&= 1\\
&\implies Y \sim N(0,1)
\end{align*}\\
(b) 
\begin{align*}
P(|X|=|Y|) &= P(|X|=|XZ|)\\
&=P(|X|=|X||Z|)\\
&= P(|X|=|X|) = 1
\end{align*}\\
(c)\\
(d) \begin{align*}
Cov[X,Y] &= E[X,Y]-E[X]E[Y]\\
&= E[XY]\\
&= E[X^2Z] = 0\\
\end{align*}
(e)\\

\noindent\textbf{Problem 8}\\
For $m,$ (temporarily substituting $y$ for $m$ to avoid confusion between $m$ and $M$ random variables), $F(y) = P(Y \leq y) = 1-P(Y> y) = 1-P(min(X_1,...,X_n) > y).$ Since all $X_i$ are i.i.d, we have $F(y) = 1-P(X_1 > y)P(X_2>y)...P(X_n>y) = 1 - P(X_1 > y)^n.$ The uniform distribution of $X_i$'s over [0,1] yields the cumulative distribution:
\begin{equation*}
F(m)=
\begin{cases}
0 & y < 0\\
1-\Big(\frac{1-m}{1}\Big)^n & m \in (0, 1)\\
1 & m>b
\end{cases}
\end{equation*}\\
The density function follows:
\begin{equation*}
f(m)=
\begin{cases}
n\Big(\frac{1-m}{1}\Big)^{n-1} & m \in [0,1]\\
0 & otherwise
\end{cases}
\end{equation*}\\
Integrating over the distribution function, $E[m]=\int_{-\infty}^{\infty}mf(m)dm = \frac{1}{n+1}$\\

\noindent\textbf{Problem 9}\\
(a) Let $S_n$ denote individual states (trials), $S_n \in {0,1}$. Each state is a Bernoulli trial with success probability $P(S_n = 1) = P(S_n = 0) = 0.5$. Define $X_n = S_1 + S_2 + ... + S_n$. For $n=1000, E[X] = (1000)(0.5) = 500, Var[X] = (1000)(0.5)(0.5) = 250.$ By the central limit theorem, $X \sim N(500,250)$. The desired probability is $P(490<X_{1000}<510)$.
\begin{align*}
P(X>490) &= 1-P(X<490)\\
&= 1-P\Bigg(\frac{X-490}{\sqrt{250}} < \frac{490-600}{\sqrt{250}}\Bigg)\\
&= 1-\Phi(-0.632)\\
&= \Phi(0.632) = 0.736\\
\end{align*}\\
By the symmetry of the normal distribution, $P(X<510) = 0.736 \implies P(490<X_{1000}<510) = 2(0.736) - 1 = 0.472$.\newline\\
(b) Chebyshev's Inequality for $X_n$, the sum of $n$ i.i.d Bernoulli trials with success probability 0.5 (defined in more detail above) gives:
\begin{equation}
P\Bigg(\left|\frac{X_n}{n} - \frac{1}{2}\right| \geq \epsilon \Bigg) \leq \frac{1}{4n\epsilon^2}
\end{equation}
Where $\epsilon$ in this case is $(0.1)(0.5) = (0.005)$. For the left hand side of (1) to be less than 0.01, $n \geq 5000$.\\

\noindent\textbf{Problem 10}\\
Proof by contradiction. Assume $\theta < 0$. Define the function $f: \mathbb{R}\to \mathbb{R}, f(x) = e^{\theta x}$. By Jensen's Inequality and convexity of $f$, $f(E[X]) \leq E[f(x)] = 1 \implies e^{\theta E[X]} \leq E[e^{\theta X}].$ For $E[X]<0$ and $\theta<0, e^{\theta E[X]} > 1 = E[e^{\theta X}]$, contradiction Jensen's Inequality for a convex function. Thus, $\theta$ must be greater than 0. 





\end{document}

