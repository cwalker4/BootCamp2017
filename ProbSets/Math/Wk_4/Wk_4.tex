\documentclass[letterpaper,12pt]{article}
\usepackage{array}
\usepackage{threeparttable}
\usepackage{geometry}
\geometry{letterpaper,tmargin=1in,bmargin=1in,lmargin=1.25in,rmargin=1.25in}
\usepackage{fancyhdr,lastpage}
\pagestyle{fancy}
\lhead{}
\chead{}
\rhead{}
\lfoot{}
\cfoot{}
\rfoot{\footnotesize\textsl{Page \thepage\ of \pageref{LastPage}}}
\renewcommand\headrulewidth{0pt}
\renewcommand\footrulewidth{0pt}
\usepackage[format=hang,font=normalsize,labelfont=bf]{caption}
\usepackage{listings}
\lstset{frame=single,
  language=Python,
  showstringspaces=false,
  columns=flexible,
  basicstyle={\small\ttfamily},
  numbers=none,
  breaklines=true,
  breakatwhitespace=true
  tabsize=3
}
\usepackage{amsmath}
\usepackage{amssymb}
\usepackage{amsthm}
\usepackage{harvard}
\usepackage{setspace}
\usepackage{float,color}
\usepackage[pdftex]{graphicx}
\usepackage{hyperref}
\hypersetup{colorlinks,linkcolor=red,urlcolor=blue}
\theoremstyle{definition}
\newtheorem{theorem}{Theorem}
\newtheorem{acknowledgement}[theorem]{Acknowledgement}
\newtheorem{algorithm}[theorem]{Algorithm}
\newtheorem{axiom}[theorem]{Axiom}
\newtheorem{case}[theorem]{Case}
\newtheorem{claim}[theorem]{Claim}
\newtheorem{conclusion}[theorem]{Conclusion}
\newtheorem{condition}[theorem]{Condition}
\newtheorem{conjecture}[theorem]{Conjecture}
\newtheorem{corollary}[theorem]{Corollary}
\newtheorem{criterion}[theorem]{Criterion}
\newtheorem{definition}[theorem]{Definition}
\newtheorem{derivation}{Derivation} % Number derivations on their own
\newtheorem{example}[theorem]{Example}
\newtheorem{exercise}[theorem]{Exercise}
\newtheorem{lemma}[theorem]{Lemma}
\newtheorem{notation}[theorem]{Notation}
\newtheorem{problem}[theorem]{Problem}
\newtheorem{proposition}{Proposition} % Number propositions on their own
\newtheorem{remark}[theorem]{Remark}
\newtheorem{solution}[theorem]{Solution}
\newtheorem{summary}[theorem]{Summary}
%\numberwithin{equation}{section}
\bibliographystyle{aer}
\newcommand\ve{\varepsilon}
\newcommand\boldline{\arrayrulewidth{1pt}\hline}


\begin{document}

\begin{flushleft}
  \textbf{\large{Math Problem Set \#4}} \\
  Charlie Walker
\end{flushleft}

\vspace{5mm}

\noindent\textbf{6.1}\\
Given $x,y \in \mathbb{R}^n, a, b, \in \mathbb{R},$ and $A \in M_n(\mathbb{R})$, choose $w \in \mathbb{R}^n$ in order to
\begin{align*}
& \text{minimize} & & -e^{-w^Tx} \\
& \text{subject to} & & w^Tx \leq w^TAw - w^TAY + a\\
& & & y^Tw = w^Tx + b
\end{align*}\\

\noindent\textbf{6.5}\\
Let $x,y \in \mathbb{R}$ be number of milk bottles and knobs, respectively. The standard form optimization problem is:
\begin{align*}
& \text{minimize} & & -(0.07x + 0.05y) \\
& \text{subject to} & & -(0.04x + 0.03y) \geq -240\\
& & & -(\frac{x}{30} + \frac{y}{60}) \geq -100
\end{align*}\\

\noindent\textbf{6.6}\\
First order conditions are:
\begin{align*}
\frac{\partial f}{\partial x} &= 6xy + 4y^2 + y = 0\\
\frac{\partial f}{\partial y} &= 8xy + 3x^2 + x = 0\\
\end{align*}
The critical points solving these equations are: $(-\frac{1}{3}, 0); (-\frac{1}{9}, -\frac{1}{12}); (0, -\frac{1}{4}); (0,0)$. The Hessian is:
\begin{equation*}
H = \begin{bmatrix}
6y & 6x + 8y + 1 \\
8y + 6x + 1 & 8x 
\end{bmatrix}
\end{equation*}
From the determinants of the Hessian evaluated at the critical points, we have that $(-\frac{1}{3}, 0)$ and $(0, -\frac{1}{4})$ are saddle points, $(-\frac{1}{9}, -\frac{1}{12})$ is a local maximum, and $(0,0)$ is inconclusive.\\

\noindent\textbf{6.11}\\
The first and second derivatives of any quadratic function $f(x) = ax^2 + bx + c$ are:
\begin{align*}
\frac{\partial f}{\partial x} &= 2ax + b\\
\frac{\partial^2 f}{\partial x^2} &= 2a
\end{align*}
The first order condition gives $x^* = \frac{-b}{2a}$. For any initial $x_0$, the Newton method gives:
\begin{equation*}
x_{k+1} = x_0 - \frac{2ax_0 + b}{2a} = \frac{-b}{2a}\\
\end{equation*}
Thus, the Newton method converges to the unique minimizer of $f$ after one iteration.

\end{document}
