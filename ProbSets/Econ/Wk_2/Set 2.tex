\documentclass[letterpaper,12pt]{article}
\usepackage{array}
\usepackage{threeparttable}
\usepackage{geometry}
\geometry{letterpaper,tmargin=1in,bmargin=1in,lmargin=1.25in,rmargin=1.25in}
\usepackage{fancyhdr,lastpage}
\pagestyle{fancy}
\lhead{}
\chead{}
\rhead{}
\lfoot{}
\cfoot{}
\rfoot{\footnotesize\textsl{Page \thepage\ of \pageref{LastPage}}}
\renewcommand\headrulewidth{0pt}
\renewcommand\footrulewidth{0pt}
\usepackage[format=hang,font=normalsize,labelfont=bf]{caption}
\usepackage{listings}
\lstset{frame=single,
  language=Python,
  showstringspaces=false,
  columns=flexible,
  basicstyle={\small\ttfamily},
  numbers=none,
  breaklines=true,
  breakatwhitespace=true
  tabsize=3
}
\usepackage{amsmath}
\usepackage{amssymb}
\usepackage{amsthm}
\usepackage{mathrsfs}
\usepackage{harvard}
\usepackage{setspace}
\usepackage{float,color}
\usepackage[pdftex]{graphicx}
\usepackage{hyperref}
\hypersetup{colorlinks,linkcolor=red,urlcolor=blue}
\theoremstyle{definition}
\newtheorem{theorem}{Theorem}
\newtheorem{acknowledgement}[theorem]{Acknowledgement}
\newtheorem{algorithm}[theorem]{Algorithm}
\newtheorem{axiom}[theorem]{Axiom}
\newtheorem{case}[theorem]{Case}
\newtheorem{claim}[theorem]{Claim}
\newtheorem{conclusion}[theorem]{Conclusion}
\newtheorem{condition}[theorem]{Condition}
\newtheorem{conjecture}[theorem]{Conjecture}
\newtheorem{corollary}[theorem]{Corollary}
\newtheorem{criterion}[theorem]{Criterion}
\newtheorem{definition}[theorem]{Definition}
\newtheorem{derivation}{Derivation} % Number derivations on their own
\newtheorem{example}[theorem]{Example}
\newtheorem{exercise}[theorem]{Exercise}
\newtheorem{lemma}[theorem]{Lemma}
\newtheorem{notation}[theorem]{Notation}
\newtheorem{problem}[theorem]{Problem}
\newtheorem{proposition}{Proposition} % Number propositions on their own
\newtheorem{remark}[theorem]{Remark}
\newtheorem{solution}[theorem]{Solution}
\newtheorem{summary}[theorem]{Summary}
%\numberwithin{equation}{section}
\bibliographystyle{aer}
\newcommand\ve{\varepsilon}
\newcommand\boldline{\arrayrulewidth{1pt}\hline}


\begin{document}

\begin{flushleft}
  \textbf{\large{Problem Set \#2, Set 2}} \\
  Charlie Walker
\end{flushleft}

\vspace{5mm}

\noindent\textbf{Exercise 1}\\
Take any $w, w' \in \mathscr{C}, x \in \mathbb{R}$. 
\begin{align*}
|Uw(x) - Uw'(x)| &\leq \rho\sup_U|\int\{w[F(x,u,z)]-w'[F(x,u,z)]\}\Phi(dz)|\\
&\leq \rho \sup_U \int|w[F(x,u,z)] - w'[F(x,u,z)]|\Phi(dz)|\\
&\leq \rho \sup_U \int\|w-w'\|\Phi(dz)\\
&= \rho \|w-w'\|
\end{align*}
Taking the sup over x finishes the proof $\implies U$ is a contraction mapping. Banach's Fixed Point Theorem implies that $U$ has one and only one fixed point $w^*$. There exists a policy function $\sigma \in \Sigma$ satisfying $Uw^* = U_\sigma w^*$. For this policy we have $w^* = Uw^* = U_\sigma w^*$. But, $v_\sigma$ is the only fixed point of $U_\sigma$, so $w^* = v_\sigma \implies w^* \leq v^*$, since $v_\sigma \leq v^*, \forall \sigma \in \Sigma$. $v_\sigma$ is thus the unique fixed point of $U$ in $\mathscr{C}$.\\

\noindent\textbf{Exercise 2}\\
See \texttt{exercises2.ipny} for solutions. 


\end{document}

